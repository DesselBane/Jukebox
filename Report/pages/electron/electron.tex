% !TEX root=../../report.tex

\Chapter{Electron}

Electron (formerly known as Atom Shell \cite{atomShell}) is an open-source framework developed and maintained by GitHub \cite{electron}. It is based on the Chromium V8 browser engine \cite{chromium} and the Node.js JavaScript runtime \cite{nodejs}. This framework was initially developed for the Atom text editor \cite{atom} in 2013 and both were opend sourced in the beginning of 2014 \cite{aboutElectron}.

Electron enables a developer to build cross-platform applications which are written in \gls{html}, \gls{css}, and \gls{js}. Its \gls{api} offers desktop integration with \gls{api}s such as the OSX and Windows notifications, the Windows taskbar, the Ubuntu launcher and many more. With this framework it is possible to maintain a single code-base while serving all major desktop operating systems as well as hosting a website on a traditional web server.

This chapter will describe the Electron framework as well as discuss which \gls{api}s are a bit more difficult to use correctly in a cross-platform environment.

\Section{Electron APIs}

Every Electron appilcation has a main file which is the root or starting point for the application. This file is specified in the package.json configuration file with the key \enquote{main}. By convention this file is called \enquote{main.js} but any name is possible.

\file{main-ready.js}{The Jukebox app \enquote{ready} event handler.}{JavaScript}

This main file is mostly used to set event handlers on the app object. These include handlers for the \enquote{ready}, \enquote{window-all-closed}, and \enquote{activate} events (see \myautoref{lst:main-ready.js}). Also the event handler for the ready event usually contains code to create a \lstcode{BrowserWindow} object and load some \gls{html}. In the Jukebox project the main.js file is used for the entire application startup sequence as well as some Electron debugging methods.

\Subsection{App}


\Subsection{Node}
\Subsection{IPC}


\addsubpage{electron/menu.tex}
\addsubpage{electron/notifications.tex}
