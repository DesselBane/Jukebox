% !TEX root=../report.tex
\Chapter{Introduction}

When developing a new application, certain aspects are to be considered. One of the main decicions is whether the application is going to be a web or a desktop application. Many applications, like the entire Microsoft Office Suite, which used to be pure desktop applications now either have web counterparts or have been moved to the web environment completely. This gives the developers access to all desktop as well as mobile platforms. Frameworks for those types of applications include Angular, jQuery, React, Vue, Ember, or Meteor for the frontend as well as \gls{asp}, Django, Express, or Ruby on Rails for the backend.

Although web applications are still gaining popularity, desktop applications have not yet ceased to exist. This may be due to the fact that web applications are subject to many restrictions. Many tasks like editing files, communicating with special hardware or querying databases are not possible. Yet these basic tasks are essential for many applications. Another reason why users might prefer desktop applications, is the fact that they can be used offline and that a desktop application is what most people are used to.

The release of the Electron Framework in 2013 opened up a way to combine the flexibility of web applications with the strength of desktop applications. Electron makes it possible to write one single code base consisting of \gls{html}, \gls{css} and \gls{js}, and still access native \gls{api}, files and databases. Electron also offers \gls{api}s for notifications or a systemtray icon, which are commonly used in desktop applications.

As a frontend framework, Angular was chosen. Angular is widely used by web developers and offers a great set of pre-built \gls{ui} elements which follow the Google Material Design guidelines. It is also very well documented and actively developed.

\gls{asp} was chosen as the server side framework since it offers great database support through \gls{ef}. It is an easy-to-learn framework which also is very well documented.

The thesis will discuss which features may not work at all and what adaptations are needed to get other features working given this specific combination of frameworks. Furthermore, the advantages and disadvantages compared to a pure desktop or web application will be analyzed.
