% !TEX root=../report.tex
\Chapter{Introduction}

When developing a new application, certain aspects are to be considered. One of the main decicions is whether the application is going to be a web or a desktop application. Many applications, like the entire Microsoft Office Suite, which used to be pure desktop applications now either have web counterparts or have been moved to the web environment completely. The web environment gives the developer access to all desktop as well as mobile platforms. Frameworks for those types of applications include Angular, jQuery, React, Vue, Ember, or Meteor for the frontend as well as \gls{asp}, Django, Express, or Ruby on Rails for the backend.

Although web applications are still gaining popularity, desktop applications have not yet ceased to exist. This may be due to the fact that web applications are subject to many restrictions. Many tasks like editing files, communicating with special hardware or querying databases are not possible. Yet these basic tasks are essential for many applications. Another reason why users might prefer desktop applications, is the fact that they can be used offline and that a desktop application is what most people are used to.

The release of the Electron Framework in 2013 opened up a way to combine the flexibility of web applications with the strength of desktop applications. Electron makes it possible to write one single code base consisting of \gls{html}, \gls{css} and \gls{js}, and still access native \glspl{api}, files and databases. Electron also offers \glspl{api} for notifications or a system tray icon, which are commonly used elements in desktop applications.

The thesis will discuss how to write an application using the Electron framework in combination with the Angular front-end framework and the \gls{asp} back-end framework. Its goal is to determine if there are features which do not work at all and what adaptations are needed to get other features working given this specific combination of frameworks. Furthermore, the advantages and disadvantages compared to a pure desktop or web application will be discussed.

\Section{The Jukebox Project}

The Jukebox project is the proof-of-concept application developed for this thesis. It is a music player. The music files are supposed to be read from disk by the \gls{asp} back-end, and then be transmitted to the Angular web front-end for playback. The \enquote{Player} refers to the code parts which are responsible for handling the playback. Another requirement was that the player can be remote controlled. For this purpose the \gls{asp} server also hosts the application as a pure web page. A second client may use this web page to control the player of the first client. For this feature websockets are used. The application is supposed to have a system tray icon when run as an Electron desktop application. Any noteworthy event should be communicated to the user via notifications.

\Subsubsection{Architecture Overview}

The entry point for the Jukebox application is in the Electron main.js file. From this file the \gls{asp} server is started as a seperate process using node's \lstcode{child\_process.spawn()} method \cite{nodeSpawn}. Then the \lstcode{BrwoserWindow} is created which will load the index.html file. This is where the Angular framework is initialized. The startup sequence is visualized in \myautoref{fig:startupSequence.png}.

\autoImg{startupSequence.png}{The Jukebox startup sequence.}

Communication between the frameworks is usually initiated from within the Angular code. When communicating with Electron \glspl{api} the \enquote{ngx-electron} package is used (\myref{subsubsec:ngxElectron}). For the communication with the \gls{asp} \gls{http} class and websockets are used.

\Subsubsection{deployment}

\Section{General stuff???}

developers can be female (duh)
recognized, separat
front-end, back-end
