% !TEX root=../report.tex
\Chapter{Introduction}

When developing a new application, certain aspects are to be considered. A fundamental decision to be made is whether the application is going to be a web or a desktop application. Many applications, like the entire Microsoft Office Suite, which used to be pure desktop applications now either have web counterparts or have been moved to the web environment completely. The web environment gives the developer access to all desktop platforms as well as mobile platforms. Frameworks for those types of applications include Angular, jQuery, React, Vue, Ember, or Meteor for the front-end as well as \gls{asp}, Django, Express, or Ruby on Rails for the back-end.

Although web applications are becoming more and more popular, desktop applications have not yet ceased to exist. This may be due to the fact that web applications are subject to many restrictions. Many tasks like editing files, communicating with special hardware or querying databases are not possible. Yet these basic tasks are essential for many applications. Other reasons why users might prefer desktop applications, is the fact that they can be used offline and that a desktop application is what most people are used to.

The release of the Electron Framework in 2013 opened up a way to combine the flexibility of web applications with the strength of desktop applications. Electron makes it possible to write one single code base consisting of \gls{html}, \gls{css} and \gls{js}, and still access native \glspl{api}, files and databases. Electron also offers \glspl{api} for notifications or a system tray icon, which are commonly used elements in desktop applications.

The thesis will discuss how to write an application using the Electron framework in combination with the Angular front-end framework and the \gls{asp} back-end framework. Its goal is to determine if there are features which do not work at all and what adaptations are needed to get other features working given this specific combination of frameworks. Furthermore, the advantages and disadvantages compared to a pure desktop or web application will be discussed.

\Subsubsection{The Jukebox Project}

The Jukebox project is the proof-of-concept application developed for this thesis. The application is a music player. The music files are read from disk by the \gls{asp} back-end, and then transmitted to the Angular web front-end for playback. The code parts which are responsible for handling the playback are referred to as the \enquote{player}. The player can be remote-controlled. For this purpose, the \gls{asp} server also hosts the application as a pure web page. A second client may use this web page to control the player created by the first client. Websockets are used to realize this feature. The application has a system tray icon when run as an Electron desktop application. Any noteworthy events are communicated to the user via notifications.

\Subsubsection{Architecture Overview}

The entry point for the Jukebox application is in the Electron main.js file. From this file, the \gls{asp} server is started as a separate process using node's \lstcode{child\_process.spawn()} method \cite{nodeSpawn}. Then the \lstcode{BrowserWindow} is created and loads the index.html file. At this point, the Angular framework is initialized. The startup sequence is visualized in \myautoref{fig:startupSequence.png}.

\autoImg{startupSequence.png}{Startup sequence of the Jukebox application.}

Communication between the frameworks is usually initiated from within the Angular code. The \enquote{ngx-electron} package is used for communication with Electron's \glspl{api}  (\myref{subsubsec:ngxElectron}). For the communication with the \gls{asp}, \gls{http} calls and websockets are used.

\Subsubsection{Deployment}

Each of the three frameworks comes with some sort of \gls{cli}, build tools, or required steps for the packaging process. For Electron, the Squirrel \cite{squirrel} installer is recommended, but during testing many problems were encountered and therefore, no installer is used. Instead, Gulp scripts \cite{gulp} are used to create a completely packaged version of the application, which then can be shipped with any installer framework or simply copied to the target system. Electron offers pre-built binaries for each operating system, these can be downloaded from the Electron site. Those binaries and all required \gls{npm} packages are copied to the target packaging location. Then, the \lstcode{electron-rebuild} \gls{npm} package is used to compile all \gls{npm} packages to target the current Electron version. After this step has been completed, the Angular \gls{cli} is used to build the Angular part of the application. This step is done twice with varying parameters. The first build targets the Electron version, and the second build targets the pure web version of the Angular part. The Angular files are then placed in the \lstcode{/resources/app/dist} folder. Next, the \gls{asp} application is published. Publishing an \gls{asp} application first compiles the source code for a given platform and then copies the binaries including all .NET Core framework libraries to the target folder. For the Jukebox project, this folder is located under \lstcode{/resources/app/api/win/} for the windows build. Finally, the \enquote{electron.exe} is renamed into \enquote{Jukebox.exe}.

\Subsubsection{General Notes}

This thesis discusses all three frameworks in a separate chapter each. All descriptions of framework methods are based on the manufacturers documentations. For the sake of simplicity and readability the text uses the male form \enquote{he} instead of \enquote{he or she} when talking about developers, this is deemed to include \enquote{she}.
