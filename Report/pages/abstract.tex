% !TEX root=../report.tex
\begin{abstract}

Many modern applications use web technologies to be able to serve a wide variety of platforms. At the same time, desktop applications offer many advantages like dedicated APIs or notifications, and they can be used offline. From a developers perspective, a single code base, which runs on all platforms is an important trait of any application. The Electron project makes it possible to meet this requirement. The user interface can be implemented with the Angular Framework. Angular is widely used by web developers and offers a great set of pre-built UI elements that follow the Google Material Design guidelines. All application data will be provided by an ASP.NET Core web server. With this combination of frameworks, an application supports Windows, Mac, and Linux plattforms as backend hosts. The Frontend can either be run as a desktop application using Electron with support for Windows, Windows UWP, Mac, and Linux or as a web application with support for all major browsers. The thesis will discuss which features may not work at all and what adaptations are needed to get others working given this specific combination of frameworks. Furthermore, the advantages and disadvantages compared to a pure desktop or web application will be analyzed.

\end{abstract}
