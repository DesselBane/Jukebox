% !TEX root=../../report.tex


\Section{Websockets}

The \gls{http} protocol was designed for web pages and is based on client initiated communication. There are techniques like long polling which simulate a two-way communication channel but even there the client needs to initialize the communication. This becomes a problem in web applications which rely on events or push notifications rather than on polling. In the Jukebox project one requirement was that a music player could be remote controlled. Websockets were used to push events like \enquote{Next} or \enquote{Play/Pause} to the player.

\file{websocket.ts}{Example websocket usage.}{JavaScript}

Websockets can be used by calling the \lstcode{WebSocket.send()} and \lstcode{WebSocket.close()} methods as well as listening to the various events. Another approach is to wrap the WebSocket into a RxJS \lstcode{Subject}, as shown in \myautoref{lst:websocket.ts} \cite{rxjs} \cite{js_websockets}. To do so the developer creates an \lstcode{Observable} which binds to the events of the WebSocket (lines \ref{line:websocket_observable_start} to \ref{line:websocket_observable_stop}). Then a \lstcode{Observer} is created (line \ref{line:websocket_observer}). In this example the \lstcode{Observer} is an object which has a \lstcode{next()} method that is empty. This is due to the fact that in the Jukebox project notifications are only received through the WebSocket. This \lstcode{next()} method is normally used to call the \lstcode{WebSocket.send()} method to send data back to the server. From the \lstcode{Observer} and \lstcode{Observable} a RxJS \lstcode{Subject} is created. This \lstcode{Subject} can then be subscribed to for incomming messages or with its \lstcode{next()} method messages can be sent.
