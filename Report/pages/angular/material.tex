% !TEX root=../../report.tex

\Section{Angular Material}

Angular Material is a framework, also developed by Google, which provides pre-built controls. These include buttons, trees, datepicker and many more. All of these controls are styled with the Google Material style guidelines in mind. To be able to use this framework the developer has to install it via \gls{npm} or a similar package management tool. \cite{angularMaterial}

\Subsection{Components}
Each control is an Angular component. This means that controls can be used in the \gls{html} by inculing their corresponding tag, as shown in \mylineref{lst:navigation-bar.html}{navbar_mat_toolbar}. Any \gls{html} tag which starts with the \lstcode{mat-} prefix is referencing an Angular Material control/component.

This prefix was changed from \lstcode{md-} to the current \lstcode{mat-}. Also with that change Angular Material changed how the components are organized into modules. It used to be a single module called \lstcode{MaterialModule} which became deprecated and replaced by many modules each for a single control \zB \lstcode{MatButtonModule}. In the Jukebox project there are currently 17 different controls in use which means 17 different modules have to be imported. To keep the app.moudle.ts file tidy a custom module was created. This so called \lstcode{MaterialMetaModule} imports and exports all 17 \enquote{submodules} so that the \lstcode{AppModule} only has to import this meta module.

\file{dialog.ts}{\revMark{Dialog example from the SettingsComponent.}}{JavaScript}

Many controls like buttons or the toolbar are only refereced in the \gls{html} since they don't contain much logic. All a developer needs from them are events or set some properties which works well via bindings (see \myref{subsubsec:bindings}). Espcially for dialogs a developer wants to have more control over how and when the dialog is displayed. Dialogs are also mostly invoked through the \gls{ts} \zB inside some method or event handler. Typically the dialog is some sort of question and its answer determines how the application should proceed. Since a dialog is a modal element which is positioned on top of everything its also not clear where in the \gls{dom} it should be placed.

Angular Material solves this by haveing the \lstcode{MatDialog} injected into the component's \gls{ts} file (\mylineref{lst:dialog.ts}{dialog_injected}). This means that the placement in the \gls{dom} is done by the framework and the developer doesn't have to worry about it. For the Jukebox project a filepicker dialog was implemented. This serves the purpose of selecting a folder, the server should use to index *.mp3 files. Once the user clicks the \enquote{addPath} button a dialog is created. This is done via the \lstcode{MatDialog.open()} method, as shown in \mylineref{lst:dialog.ts}{dialog_open}. The \lstcode{MatDialog.open()} method \revMark{requires the type of an Angular component}. This is the component to be displayed by the dialog. Further a configuration object may be passed to the method. The configuration object is of type \lstcode{MatDialogConfig} and was used to pass a custom config object of type \lstcode{FilePickerConfig} (\mylineref{lst:dialog.ts}{dialog_filePickerConfig}).

The \lstcode{MatDialog.open()} method returns a reference to the newly created dialog which is of type \lstcode{MatDialogRef<T>}. This class exposes certain methods which configure event handlers. Most notably the \lstcode{afterClosed()} and \lstcode{afterOpen()} methods. These methods return \lstcode{Observable<R | undefined>} and \lstcode{Observable<void>} respectively. The \lstcode{afterClosed()} method was used to retrieve the selected path from the \lstcode{FilePickerConfig} object and initiate an \gls{http} call to save this path.

\file{filePickerComponent.ts}{Excerpt from the \lstcode{FilePickerDialogComponent}.}{JavaScript}

Once the \lstcode{MatDialog.open()} method is called the required component will be created. During that process all necessary objects will be injected into this component. At this point the developer can request dialog specific objects next to the service classes the component needs to function. A reference of the displaying dialog is in almost any case necessary. For this the developer asks for an object of type \lstcode{MatDialogRef<T>} where \lstcode{T} is the type of the displayed component (\mylineref{lst:filePickerComponent.ts}{filePicker_dialogRef}). With this reference it is possible to close the dialog from within the \gls{ts} code, as shown in line \ref{line:filePicker_close}. If a configuration object was passed its data can be accessed via the \lstcode{@Inject()} decorator (\mylineref{lst:filePickerComponent.ts}{filePicker_config}). For this to work the \lstcode{MAT\_DIALOG\_DATA} argument must be passed into the decorator.

\cite{matDialog}


\Subsection{Icons}

Fitting icons are a big part of the visual design of any application. There are many ways how icons can be encoded and used \zB they can be images or vector graphics. Angular Material uses another also popular method called \enquote{Font Icons}. The same concept is used by FontAwesome \cite{fontAwesome} and many more. A font usually displays some sort of letter but technically it is just a way of reprecenting some scalable image. Icons are mostly used in buttons or menus which are all designed to work with some sort of font. Also fonts are an old concept which means support for them is fairly well.

To use the Material icon set there are many ways. The Jukebox project mainly uses the \lstcode{<mat-icon />} tag (see \mylineref{lst:navigation-bar.html}{navbar_icon}) but it is also possible to associate \gls{css} classes with specific icons or use the \lstcode{<svg />} tag. It should be noted though, that the latter two examples are usually needed if the provided icons aren't sufficient or if \zB another icon font is used in addition. Also for the font to work it has to be included in the website which was done via the \lstcode{@font-face} \gls{css} rule.

During the development of the Jukebox project a bug was discovered which resulted in no icons beeing shown. This only appeared in the Internet Explorer and Edge and was fixed by changing how an icon is selected. Angular Material offers the possibility to use an icons name to referece it inside the \gls{html} tag \zB \lstcode{<mat-icon>menu</mat-icon>}. Another method is to use the Unicode explicitly. This has the drawback that it isn't obvious anymore what an icon represents or looks like by just looking at the \gls{html} code. Some users in the community have created command line tools, which integrate into the Angular \gls{cli} or work as standalone scripts, which replace all names with their corresponding Unicodes. Since the focus for this thesis was towards compatibility between Electron and Angular none of these tools were evaluated. \cite{materialIcons}
