% Shortcuts with correct space
\newcommand{\ua}{\mbox{u.\,a.\ }}
\newcommand{\idr}{\mbox{i.\,d.\,R. }}
\newcommand{\ca}{\mbox{ca.\ }}
\newcommand{\zB}{\mbox{z.\,B.\ }}
\newcommand{\dahe}{\mbox{d.\,h.\ }}
\newcommand{\vgl}{Vgl.\ }
\newcommand{\bzw}{bzw.\ }
\newcommand{\evtl}{evtl.\ }
\newcommand{\dw}{des Weiteren}
\newcommand{\uu}{u.\ U.\ }

% Enumeration
\newcommand{\enumDot}[1]{
	\begin{itemize}
		#1
	\end{itemize}
}

\newcommand{\enumNum}[1]{
	\begin{enumerate}
		#1
	\end{enumerate}
}

% Zoning
\newcommand{\Chapter}[2][]{
	\chapter{#2}
	\ifthenelse{\equal{#1}{}}{}{\label{ch:#1}}
}

\newcommand{\Section}[2][]{
	\section{#2}
	\ifthenelse{\equal{#1}{}}{}{\label{sec:#1}}
}

\newcommand{\Subsection}[2][]{
	\subsection{#2}
	\ifthenelse{\equal{#1}{}}{}{\label{subsec:#1}}
}

\newcommand{\Subsubsection}[2][]{
	\subsubsection{#2}
	\ifthenelse{\equal{#1}{}}{}{\label{subsubsec:#1}}
}

% Header and Footer
\renewcommand*{\chapterpagestyle}{scrheadings}
\renewcommand{\headfont}{\normalfont}

% tableheader with: name, span of 1, center, and bold
\newcommand{\thead}[1]{ \multicolumn{1}{c}{\bfseries #1} }

% references and inputs
\newcommand{\file}[3]{
	\lstinputlisting [caption={#2},label={lst:#1}, language=#3, escapechar=|]{ res/code/#1 }
}

\newcommand{\fileJS}[3]{
	\lstinputlisting [caption={#2},label={lst:#1}, language=#3, escapechar=|, style=htmlcssjs]{ res/code/#1 }
}

\newcommand{\fileSegment}[5]{
	\lstinputlisting [caption={#2},label={lst:#1}, language=#3,firstline={#4},lastline={#5}]{ res/code/#1 }
}

\newcommand{\showimg}[2][scale=1.0]{ \includegraphics[{#1}]{res/images/#2} }
\newcommand{\image}[3][scale=1.0]{
	\begin{figure}[ht]
		\centering
		\showimg[#1]{#2}
		\caption{#3}
		\label{fig:#2}
	\end{figure}
}

\newcommand{\autoImg}[2]{
	\begin{figure}[ht]
		\centering
		\includegraphics[width=1.0\textwidth]{res/images/#1}
		\caption{#2}
		\label{fig:#1}
	\end{figure}
}

\newcommand{\tbl}[3]{
	\begin{table}[h]
		\begin{center}
			%\rowcolors{startrow}{color}{color}
			\begin{tabular}{#2}
				#3
			\end{tabular}
			\caption{#1}
			\label{table:{#1}}
		\end{center}
	\end{table}
}
% add glossary entry

\newcommand{\entry}[2]{\newglossaryentry{#1}{name={#1},description={#2}}}

% einfaches Wechseln der Schrift, z.B.: \changefont{cmss}{sbc}{n}
\newcommand{\changefont}[3]{\fontfamily{#1} \fontseries{#2} \fontshape{#3} \selectfont}
\newcommand{\bs}{$\backslash$}

% zum Ausgeben von Autoren
\newcommand{\AutorName}[1]{\textsc{#1}}
\newcommand{\Autor}[1]{\AutorName{\citeauthor{#1}}}

% verschiedene Befehle um Wörter semantisch auszuzeichnen ----------------------
\newcommand{\NeuerBegriff}[1]{\textbf{#1}}
\newcommand{\Fachbegriff}[1]{\textit{#1}}
\newcommand{\Eingabe}[1]{\texttt{#1}}
\newcommand{\Code}[1]{\texttt{#1}}
\newcommand{\Datei}[1]{\texttt{#1}}
\newcommand{\Datentyp}[1]{\textsf{#1}}
\newcommand{\XMLElement}[1]{\textsf{#1}}
\newcommand{\Webservice}[1]{\textsf{#1}}

\newcommand{\addpage}[1]{\include{pages/#1}}
\newcommand{\addsubpage}[1]{\input{pages/#1}}

%hyperref
% Befehle, die Umlaute ausgeben, führen zu Fehlern, wenn sie hyperref als
% Optionen übergeben werden
\hypersetup{
	pdftitle={\titel \untertitel},
	pdfauthor={\autor},
	pdfcreator={\autor},
	pdfsubject={\titel \untertitel},
	pdfkeywords={\titel \untertitel}
}

\newcommand*{\myautoref}[1]{\hyperref[{#1}]{\autoref{#1} \vpageref{#1}}}
\newcommand*{\myref}[1]{\hyperref[{#1}]{\autoref{#1} \nameref{#1} \vpageref{#1}}}
\newcommand*{\mylineref}[2]{\hyperref[{#1}]{\autoref{#1} \vpageref{#1}, Zeile \ref{line:#2}}}
\newcommand*{\mynameref}[1]{\hyperref[{#1}]{\nameref{#1} \vpageref{#1}}}

\newcommand{\lstcode}[1]{\texttt{#1}}
\newcommand{\revMark}[1]{\textcolor{red}{#1} \ref{revMarkerLABEL}}
