% !TEX root=../report.tex
% language packages
\usepackage[ngerman]{babel}
\usepackage[utf8]{inputenc}
\usepackage[T1]{fontenc}
\usepackage[dvips,final]{graphicx} % graphics package (jpg possible)
%\usepackage[table]{
%	xcolor     % color package (red, green, blue, yellow, cyan, magenta, black, white)
%}


\usepackage[
style=alphabetic,
bibencoding=latin1,
hyperref=true,
alldates=long,
block=par
]{biblatex}
\usepackage{
	csquotes,		% quotes
	textcomp,   % Euro-Zeichen etc.
	lmodern,    %
	relsize,    % font package
	amsmath,    % AMSTeX math symbols package (e. g. \boldsymbol \mathbb)
	amsfonts,   %
	setspace,   % linespace package
	geometry,   %
	floatflt,   % image floating package
	url,        % URL package
	imakeidx,   % index package (mark word with \index{category!word})
	chngcntr,   % numbering footnotes package
	paralist,   % list format package
	ifthen,     % ifthen package
	xspace,     % remove whitespace in macros
	%longtable,  %
	array,      %
	ragged2e,   %
	%lscape,     % long tables
	listings,    % listings package
	titlesec,
	varioref,
	soul,
	alltt,
	xcolor,
	enumitem
}


\usepackage[
automark,     % Kapitelangaben in Kopfzeile automatisch erstellen
headsepline,  % Trennlinie unter Kopfzeile
ilines        % Trennlinie linksbündig ausrichten
]{scrpage2}

% ------------------------------------------------------------------------------
% PDF-Optionen
% ------------------------------------------------------------------------------
\usepackage[
bookmarks,
bookmarksopen=true,
colorlinks=true,
% diese Farbdefinitionen zeichnen Links im PDF farblich aus
linkcolor=blue, % einfache interne Verknüpfungen
anchorcolor=black,% Ankertext
citecolor=blue, % Verweise auf Literaturverzeichniseinträge im Text
filecolor=magenta, % Verknüpfungen, die lokale Dateien öffnen
menucolor=red, % Acrobat-Menüpunkte
urlcolor=cyan,
% diese Farbdefinitionen sollten für den Druck verwendet werden (alles schwarz)
%linkcolor=black, % einfache interne Verknüpfungen
%anchorcolor=black, % Ankertext
%citecolor=black, % Verweise auf Literaturverzeichniseinträge im Text
%filecolor=black, % Verknüpfungen, die lokale Dateien öffnen
%menucolor=black, % Acrobat-Menüpunkte
%urlcolor=black,
%backref, % backref from library to pages
plainpages=false, % zur korrekten Erstellung der Bookmarks
pdfpagelabels, % zur korrekten Erstellung der Bookmarks
hypertexnames=false, % zur korrekten Erstellung der Bookmarks
linktocpage % Seitenzahlen anstatt Text im Inhaltsverzeichnis verlinken
]{hyperref}

% use package after hyperref
\usepackage[nonumberlist,nopostdot]{glossaries}
